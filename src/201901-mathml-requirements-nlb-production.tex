% !TeX spellcheck = nb_NO
\documentclass[english,a4paper,11pt]{article}
\usepackage{babel}
\usepackage{bookman}
\usepackage[utf8]{inputenx}
\usepackage[T1]{fontenc}
\usepackage{xspace}
\usepackage{lastpage}
\usepackage{fancyhdr}
\usepackage{tabularx}
\usepackage{tikz}
\usepackage[normalem]{ulem}

\usepackage{varioref}
\usepackage{url}
\usepackage{moreverb}
\usepackage{ifthen}
\usepackage{amsmath}
\usepackage{siunitx}
%\usepackage{esint}
%\usepackage{mathdesign}
\usepackage{color}
\usepackage[a4paper,left=15mm,right=30mm,top=25mm,bottom=25mm]{geometry}

\usetikzlibrary{trees,arrows,arrows.meta,intersections,calc,backgrounds,positioning,decorations} % ,shadows,fit,shapes,backgrounds,decorations

\setlength{\parskip}{1ex plus 4pt minus 2pt}
\setlength{\marginparwidth}{18mm}
%\setlength{\marginparsep}{3pt}
\input{def-mathml-req.tex}


\title{Requirements for MathML markup\\ in EPUB files for NLB}
\author{Per Sennels, NLB}

\begin{document}
	\bunntekst
	\thispagestyle{empty}
	\raggedright
	
	\maketitle
	\thispagestyle{empty}
	\vfill
	\begin{quote}
		{\bfseries This is a very early version of a description of NLB's requirements for MathML markup in EPUB files.
		\smallskip
		
		Note that this is a work in progress; the scope of mathematical topics to be covered will \innholdEndret{20190130}{certainly}{probably} be widened in the future. }
	
		\bigskip
		\emph{Version history:}\\
		\textbullet\quad20181220 / 1.0: First version\\
		\textbullet\quad20190130 / 1.1: Second version (this version): Many new topics added, and there are smaller edits of the original text. \textcolor{endretInnhold}{Changes and additions printed in dark red, and with date of change in margin.}
	\end{quote}
	\vfill
	\pagebreak
	\tableofcontents
%	\vfill
	\pagebreak

\section{The purpose of MathML in NLB's EPUB files}

NLB will use MathML in the EPUB files to present mathematical expressions in a variety of ways in the distributed versions of the content:
\begin{itemize}
	\item In a talking book version, we may synchronize a narrated version of the expression to the standard mathematical notation, the latter being generated automatically from MathML by the browser displaying the math as well as the text content of the talking book.

	\item In an e-book version we may present a MathML expression in the normal way; as a standard mathematical expression, using standard mathematical notation. In addition, we may present an automatically generated  textual representation of the mathematical expression, as an alternative or a supplement to those who prefer to use local synthetic speech or refreshable Braille to digest the content.
	
	
	\item In a TTS based talking book version, we may automatically generate a string that represents a verbal interpretation of the mathML. This string can then be used as a basis for TTS generation of an audio segment that represents the mathematical expression.
	
	\item The MathML version of a mathematical expression can be used as a basis for printed Braille. Once again, the MathML may be converted to text by XML processing, and the text version may then be refined and converted to Braille.
\end{itemize}
Except for the first bullet point, which relies on the math knowledge of a  human narrator, production of the distributed version involve some kind of automatic transformation of the MathML markup into some other kind of textual representation, typically a pure text string containing a verbal representation of the math in question. 

\begin{center}
\begin{tikzpicture}
\node[text width=40mm,align=center,draw=black!30] (matte) {In printed book:\\$x =\sqrt{2}$};

\node[text width=40mm,align=center,draw=black!30,right=25mm of matte] (MathML) {Markup:
\begin{kodeblokk}
\footnotesize
\begin{verbatimtab}[3]
<math ...>
	<mi>x</mi>
	<mo>=</mo>
	<msqrt>
		<mn>2</mn>
	</msqrt>
</math>
\end{verbatimtab}
\end{kodeblokk}
};

\node[text width=30mm,align=center,draw=black!30,right=25mm of MathML] (xslt) {XSLT \mbox{transformation}};
\node[text width=45mm,align=center,draw=black!30,below=25mm of xslt] (tale) {Verbal \mbox{representation}\\\verbalt{x equals square root of two}};

\draw[-latex',thick] (matte) -- (MathML);
\draw[-latex',thick] (MathML) -- (xslt);
\draw[-latex',thick] (xslt) -- (tale);
\end{tikzpicture}
\end{center}
The quality of the generated text string relies heavily on the quality of the MathML markup, and care must be taken to create MathML markup that~\ldots
\begin{itemize}
	\item \ldots correctly represents the printed mathematical expressions;
	\item \ldots is compliant with the W3C recommendations for use of MathML markup;
	\item \ldots respects the NLB specific requirements stated in this document;
	\item \ldots ensures that the mathematical information that can be detected from the markup, is as unambiguous as possible.
\end{itemize}
The last point is very important, as the interpretation of a mathematical expression often relies on the context  the expression is placed in, and also on the interpreter's (e.g. the student's) understanding of that context.

As an example, we can investigate the following expression:
\[
a (t + \varphi)
\]
One interpretation of this expression is that it represents a variable, $a$, that is to be multiplied by the sum of two other variables, $t + \varphi$.
 
A completely different interpretation is that the expression represents a function, $a$, with one argument, namely the sum of $t$ and $\varphi$.

It is extremely important that the interpretation is clearly indicated in the MathML markup. If the expression represents a multiplication operation, this should be clarified by use of the MathML operator
\verb|<mo>&InvisibleTimes;</mo>|. And similar, the representation of a function must be clarified by using
\verb|<mo>&ApplyFunction;</mo>|.


So, even though this:
\begin{kodeblokk}
	\begin{verbatimtab}[3]
<math xmlns="http://www.w3.org/1998/Math/MathML" display="block">
	<mi>a</mi>
	<mo>(</mo>
	<mi>t</mi>
	<mo>+</mo>
	<mi>&phi;</mi>
	<mo>)</mo>
</math>
\end{verbatimtab}
\end{kodeblokk}
in many cases would be perfectly good markup of the expression above, NLB require the markup to be either
\begin{kodeblokk}
	\begin{listing}{1}
<math xmlns="http://www.w3.org/1998/Math/MathML" display="block">
	<mi>a</mi>
	<mo>&InvisibleTimes;</mo>
	<mfenced>
		<mrow>
			<mi>t</mi>
			<mo>+</mo>
			<mi>&phi;</mi>
		</mrow>
	</mfenced>
</math>
\end{listing}
\end{kodeblokk}
for the ''multiplication interpretation'', or
\begin{kodeblokk}
	\begin{listing}{1}
<math xmlns="http://www.w3.org/1998/Math/MathML" display="block">
	<mi>a</mi>
	<mo>&ApplyFunction;</mo>
	<mfenced>
		<mrow>
			<mi>t</mi>
			<mo>+</mo>
			<mi>&phi;</mi>
		</mrow>
	</mfenced>
</math>\end{listing}
\end{kodeblokk}
for the ''function interpretation''. Note that the only difference is the choice of entity in line~3.

It should be quite clear by now that, to ensure the \innholdEndret{20190130}{proper}{expected} quality of the MathML markup, personnel with solid mathematical skills, as well as the ability to focus on important markup details, must be assigned to this kind of work.


%If the expression above stands completely alone, it is not possible to decide the correct representation. However, if there is an overlaying context, perhaps if the expression above is part of a larger mathematical expression, the representation should be clear enough.
%
%This equation
%\[
%a (t + \varphi) = a \cdot t + a \cdot \varphi
%\]
%clearly indicates that we are talking about multiplication, while
%\[
%a (t + \varphi) = \frac{\partial^2 x(t + \varphi)}{\partial t^2}
%\]
%indicates that we are talking about functions, perhaps related to acceleration and position.

\bigskip

This document aims to specify the exact markup we want, when there seems to be different ways to use MathML to represent the mathematical expression. Note that this is a work in progress; the scope of mathematical topics to be covered will certainly be widened in the future. You should also expect changes in established requirements as we gain experience with the automatic transformation of MathML to alternative formats.


%\todo{Skrive om hva vi skal bruke markupen til}
%\todo{Skrive om MathML kan gjøres på flere måter, og at vi krever at det gjøres på én spesiell måte.}
%\todo{Skrive om at dette kan bli endret og utvidet}

\section{The fundamental requirements}

\begin{itemize}
	\item All mathematical expressions --~both inline and block~-- shall be marked up, along with the general EPUB markup, using \emph{MathML Presentation Markup}, as spesified in ''Mathematical Markup Language (MathML) Version 3.0 2nd Edition''\footnote{See~\url{https://www.w3.org/TR/MathML3/}}.
	\item Whenever applicable, the requirements in this document must be respected. 
	
	For mathematical expressions that are not covered in this document, the personnel involved in the markup process is encouraged to do MathML markup based on a good understanding of mathematics, combined with a solid knowledge of the set of MathML elements and attributes.
%
%	\item Unless MathML markup of a certain type of mathematical expression is specified in this document, the markup should be based on a good understanding of mathematics, combined with a solid knowledge of the set of MathML elements and attributes.
	\item All MathML markup must be annotated with an AsciiMath expression that represents the mathematical expression printed in the book. Thus, the complete markup of any given mathematical expression must be coded according to the following scheme:
	
\begin{kodeblokk}
\begin{verbatimtab}[3]
<math xmlns="http://www.w3.org/1998/Math/MathML" display="[value]">
	<semantics>
		<mrow>
			[MathML markup]
		</mrow>
		<annotation encoding="ASCIIMath">
			[AsciiMath markup]
		</annotation>
	</semantics>
</math>
\end{verbatimtab}
\end{kodeblokk}
The value of the \markup{display} attribute shall always be either \markup{block} or \markup{inline}, depending on the placement of the expression in the printed book. The \markup{[MathML markup]} is, of course, the MathML markup that represents the mathematical expression in question, and \markup{[AsciiMath markup]} is the AciiMath version of the same expression.

Observe that the MathML markup will always be a child of an \element{mrow} element, which again is a child of an \element{semantics} element, which again is a child of an \element{math} element, which constitute the frame around the complete MathML markup.

\bigskip
Please consult \url{https://www.w3.org/TR/MathML3/chapter5.html#mixing.semantic.annotations} for further information about annotation of MathML markup.
\end{itemize}

\bigskip

Please note that, in all of the examples to follow, the markup discussed is the one represented by the \markup{[MathML markup]} part. The outer \element{math}, \element{semantics} and \element{mrow} framework is omitted, to put focus on the matter in question. These elements are of course always required as a container for each of the mathematical expressions in the EPUB file.


\section{Additional requirements and examples}

\subsection{Use of invisible operators}

If there is any risk of ambiguity, the following operators \textbf{must} be used as entities in the markup:

\begin{tabular}{llll}
\multicolumn{1}{c}{\textbf{Entity}}
	& \multicolumn{1}{c}{\textbf{Short}}
		& \multicolumn{1}{c}{\textbf{Numeric}}
			& \multicolumn{1}{c}{\textbf{Comment}}\\
\entitet{ApplyFunction} & \entitet{af} & \entitet{\#8289} & Function application\\
\entitet{InvisibleTimes} & \entitet{it} & \entitet{\#8290} & Invisible multiplcation\\
\entitet{InvisibleComma} & \entitet{ic} & \entitet{\#8291} & Invisible separator\\	% bruker vi denne?
 & & \entitet{\#8292} & Invisible addition
\end{tabular}

\innholdNyttBlokkStart{20190130}%
This means that, even for expressions such as $(x+y)\, (x-y)$ or $2\,\sin \alpha$, where it is quite clear from the context that multiplication is involved, we \emph{require} that these multiplications are added to the markup. Thus, the first of these two expressions must be marked up as 

\begin{kodeblokk}
	\begin{listing}{1}
<mfenced>
	<mrow>
		<mi>x</mi>
		<mo>+</mo>
		<mi>y</mi>
	</mrow>
</mfenced>
<mo>&InvisibleTimes;</mo>
<mfenced>
	<mrow>
		<mi>x</mi>
		<mo>-</mo>
		<mi>y</mi>
	</mrow>
</mfenced>\end{listing}
\end{kodeblokk}
while the second expression must be represented with the following markup:
\begin{kodeblokk}
	\begin{listing}{1}
<mn>2</mn>
<mo>&#8290;</mo>
<mrow>
	<mi>sin</mi>
	<mo>&#8289;</mo>
	<mi>&alpha;</mi>
</mrow>\end{listing}
\end{kodeblokk}
Note the use of \entitet{InvisibleTimes} in line~8 and line~2 respectively in these markup snippets.\innholdNyttBlokkSlutt

Multiple examples of use \innholdNyttInline{20190130}{of invisible operators} are given in the following sections.

\subsection{Markup of parentheses}

Parentheses shall not be marked up using \verb|<mo>(</mo>| and \verb|<mo>)</mo>|. Rather the \innholdNyttInline{20190130}{MathML element} \element{mfenced} \innholdSlettet{20190130}{element} must be used.

This is trivial when there is only one element inside the parentheses, such as $g(x)$, which could be marked up as
\begin{kodeblokk}
\begin{verbatimtab}[3]
<mrow>
	<mi>g</mi>
	<mo>&ApplyFunction;</mo>
	<mfenced>
		<mi>x</mi>
	</mfenced>
</morw>
\end{verbatimtab}
\end{kodeblokk}

However, when the content of the parentheses consists of multiple parts, such as $3 \cdot (4 + 9)$, the content of the \element{mfenced} element must be placed inside an \element{mrow} element:
\begin{kodeblokk}
\begin{verbatimtab}[3]
<mn>3</mn>
<mo>&sdot;</mo>
<mfenced>
	<mrow>
		<mn>4</mn>
		<mo>+</mo>
		<mn>9</mn>
	</mrow>
</mfenced>	
\end{verbatimtab}
\end{kodeblokk}
A more complex example, involving nested parentheses, is
\[
4 \cdot (x + x \cdot (x +a)) \cdot (x - x\cdot (x -b))
\]
The correct markup of this expression would be
\begin{kodeblokk}
	\begin{verbatimtab}[3]
<mn>4</mn>
<mo>&sdot;</mo>
<mfenced>
	<mrow>
		<mi>x</mi>
		<mo>+</mo>
		<mi>x</mi>
		<mo>&sdot;</mo>
		<mfenced>
			<mrow>
				<mi>x</mi>
				<mo>+</mo>
				<mi>a</mi>
			</mrow>
		</mfenced>
	</mrow>
</mfenced>
<mo>&sdot;</mo>
<mfenced>
	<mrow>
	<mi>x</mi>
	<mo>-</mo>
	<mi>x</mi>
	<mo>&sdot;</mo>
	<mfenced>
		<mrow>
			<mi>x</mi>
			<mo>-</mo>
			<mi>b</mi>
		</mrow>
	</mfenced>
	</mrow>
</mfenced>
	\end{verbatimtab}
\end{kodeblokk}


\bigskip
Consult \url{https://www.w3.org/TR/MathML3/chapter3.html#presm.mfenced} for further information about use of the \element{mfenced} element.

\innholdNyttBlokkStart{20190130}
\subsection{Markup of the absolute value}
The absolute value of a number, symbol or expression shall be done using the MathML element \element{mfenced} together with the \markup{open} and \markup{close} attributes. Both attributes must have the value \markup{|}.

The expression 
\[|-2| = 2\]
must be marked up as
\begin{kodeblokk}
	\begin{verbatimtab}[3]
<mfenced open="|" close="|">
	<mrow>
		<mo>-</mo>
		<mn>2</mn>
	</mrow>
</mfenced>
<mo>=</mo>
<mn>2</mn>
\end{verbatimtab}
\end{kodeblokk}
while the more complex expression
\[
\left|\cos \frac{x}{2}\right| =\sqrt{\frac{1 + \cos x}{2}}
\]
must be marked up as follows:
\begin{kodeblokk}
	\begin{verbatimtab}[3]
<mfenced open="|" close="|">
	<mrow>
		<mi>cos</mi>
		<mo>&#8289;</mo>
		<mfrac>
			<mi>x</mi>
			<mn>2</mn>
		</mfrac>
	</mrow>
</mfenced>
<mo>=</mo>
<msqrt>
	<mrow>
		<mfrac>
			<mrow>
				<mn>1</mn>
				<mo>+</mo>
				<mrow>
					<mi>cos</mi>
					<mo>&#8289;</mo>
					<mi>x</mi>
				</mrow>
			</mrow>
			<mn>2</mn>
		</mfrac>
	</mrow>
</msqrt>
\end{verbatimtab}
\end{kodeblokk}

\subsection{Markup of number of degrees}

How to mark up a number of degrees, such as such as $\ang{360}$ or $\ang{-273,15}\;\text{C}$, depends on whether the value is positive or negative.

For a positive number, the required markup is simple:
\begin{kodeblokk}
\begin{verbatimtab}[3]
<mrow>
	<mn>[numeric value]</mn>
	<mi>&#176;</mi>
</mrow>
\end{verbatimtab}
\end{kodeblokk}

\kravliste{
	\item The markup must be placed inside an \element{mrow} element. This \element{mrow} must contain exactly two children.
	\item The first child of the \element{mrow} element must be an \element{mn} element, containing the relevant value.
	\item The second child of the \element{mrow} must be an \element{mo} element, containing the degree symbol, represented by the numeric entity \numentitet{176}.
}
Note also that the \element{msup} element must not be used, as the degree symbol by itself represents a raised ring.

\medskip
For a negative value, the required markup is a bit more complex:
\begin{kodeblokk}
\begin{verbatimtab}[3]
<mrow>
	<mrow>
		<mo>-</mo>
		<mn>[numeric value]</mn>
	</mrow>
	<mi>&#176;</mi>
</mrow>
\end{verbatimtab}
\end{kodeblokk}

\kravliste{
	\item Once again the markup must be placed inside an \element{mrow} element with exactly two children.
	\item However, this time the first child of the \element{mrow} element must be another \element{mrow} element:
	\begin{itemize}
		\item The first child of this \element{mrow} element must be an \element{mo} element containing the normal hyphen sign.
		\item The second child of this \element{mrow} element must be an \element{mn} element, containing the relevant absolute value.
	\end{itemize}
	\item The second child of the containing \element{mrow} element must be an \element{mo} element, containing the degree symbol, represented by the numeric entity \numentitet{176}.
}

\medskip
This kind of markup is relevant both for  temperature and for angular measurements. But for angular measurements there is --~in addition to the normal 360 degree division of a circle~-- the \emph{gradian measure} where the circle is divided into 400~gon, so that $\ang{360} = 400^\text{g}$.

The required markup for this kind of angular measurement is either:
\begin{kodeblokk}
\begin{verbatimtab}[3]
<msup>
	<mn>[numeric value]</mn>
	<mtext>g</mtext>
</msup>
\end{verbatimtab}
\end{kodeblokk}
or
\begin{kodeblokk}
\begin{verbatimtab}[3]
<msup>
	<mrow>
		<mo>-</mo>
		<mn>[numeric value]</mn>
	</mrow>
	<mtext>g</mtext>
</msup>
\end{verbatimtab}
\end{kodeblokk}
depending on the value, as described above.

Note the use of the \element{msup} element this time, in order to raise the unit g.

\begin{eksempler}
The markup for the expression
\[
\ang{0}\;\text{C} = \ang{32}\;\text{F} = 273\;\text{K}
\]	
is
\begin{kodeblokk}
\begin{verbatimtab}[3]
<mrow>
	<mn>0</mn>
	<mi>&#176;</mi>
</mrow>
<mspace width="0.25em"/>
<mtext>C</mtext>
<mo>=</mo>
<mrow>
	<mn>32</mn>
	<mi>&#176;</mi>
</mrow>
<mspace width="0.25em"/>
<mtext>F</mtext>
<mo>=</mo>
<mn>273</mn>
<mspace width="0.25em"/>
<mtext>K</mtext>
\end{verbatimtab}
\end{kodeblokk}
Note the use of the \element{mspace} element to insert proper visual spacing between values and the relevant unit.

For clarity, one could use the following markup instead:
\begin{kodeblokk}
\begin{verbatimtab}[3]
<mrow>
	<mrow>
		<mn>0</mn>
		<mi>&#176;</mi>
	</mrow>
	<mspace width="0.25em"/>
	<mtext>C</mtext>
</mrow>
<mo>=</mo>
<mrow>
	<mrow>
		<mn>32</mn>
		<mi>&#176;</mi>
	</mrow>
	<mspace width="0.25em"/>
	<mtext>F</mtext>
</mrow>
<mrow>
	<mo>=</mo>
	<mn>273</mn>
	<mspace width="0.25em"/>
	<mtext>K</mtext>
</mrow>
\end{verbatimtab}
\end{kodeblokk}
The only difference is the extra \element{mrow} elements that are used to group together the different parts of the expression.

The expression $T_0 = \ang{-273,15}\;\text{C}$ must be represented by the following markup:
\begin{kodeblokk}
\begin{verbatimtab}[3]
<msub>
	<mi>T</mi>
	<mn>0</mn>
</msub>
<mo>=</mo>
<mrow>
	<mrow>
		<mo>-</mo>
		<mn>273,15</mn>
	</mrow>
	<mi>&#176;</mi>
</mrow>
<mspace width="0.25em"/>
<mtext>C</mtext>
\end{verbatimtab}
\end{kodeblokk}

As a final example, the expression \[
\sin \ang{45} =\sin 50^\text{g}  = \sin\frac{\pi}{2}= \frac{1}{\sqrt{2}}
\]
must be marked up as
\begin{kodeblokk}
	\begin{verbatimtab}[3]
<mrow>
	<mi>sin</mi>
	<mo>&af;</mo>
	<mrow>
		<mn>45</mn>
		<mi>&#176;</mi>
	</mrow>
</mrow>
<mo>=</mo>
<mrow>
	<mi>sin</mi>
	<mo>&af;</mo>
	<msup>
		<mn>50</mn>
		<mtext>g</mtext>
	</msup>
</mrow>
<mo>=</mo>
<mrow>
	<mi>sin</mi>
	<mo>&af;</mo>
	<mfrac>
		<mi>&pi;</mi>
		<mn>4</mn>
	</mfrac>
</mrow>
<mo>=</mo>
<mfrac>
	<mn>1</mn>
	<msqrt>
		<mn>2</mn>
	</msqrt>
</mfrac>
	\end{verbatimtab}
\end{kodeblokk}

\end{eksempler}

\subsection{Markup of square roots and higher-order roots}
The square root must be marked up using the MathML element \element{msqrt}. As the square root of a mathematical expression can be looked upon as a function with \emph{one} argument, one could expect that the \element{sqrt} element always should have \emph{exactly on}e child.
This would require that $\sqrt{a + b}$ should be marked up as
\begin{kodeblokk}
\begin{verbatimtab}[3]
<msqrt>
	<mrow>
		<mi>a</mi>
		<mo>+</mo>
		<mi>b</mi>
	</mrow>
</msqrt>
\end{verbatimtab}
\end{kodeblokk}
And this is indeed perfectly good markup, but the simpler form
\begin{kodeblokk}
\begin{verbatimtab}[3]
<msqrt>
	<mi>a</mi>
	<mo>+</mo>
	<mi>b</mi>
</msqrt>
\end{verbatimtab}
\end{kodeblokk}
would work just as well.

\bigskip For root expressions other than the square root, the MathML element \element{mroot} must be used, and this time with the requirement that only two children are allowed. The first child must be the expression of which one wants to find the radical, and the second child must be the order of the root.

Thus, the expression \[\sqrt[3]{8} =2\]
must be marked up as
\begin{kodeblokk}
\begin{verbatimtab}[3]
<mroot>
	<mn>8</mn>
	<mn>3</mn>
</mroot>
<mo>=</mo>
<mn>2</mn>
\end{verbatimtab}
\end{kodeblokk}
and the expression \[\sqrt[n]{x}=x^{1/n} \]
must be marked up as
\begin{kodeblokk}
\begin{verbatimtab}[3]
<mroot>
	<mi>x</mi>
	<mn>n</mn>
</mroot>
<mo>=</mo>
<msup>
	<mi>x</mi>
	<mfrac bevelled="true">
		<mn>1</mn>
		<mi>n</mi>
	</mfrac>
</msup>
\end{verbatimtab}
\end{kodeblokk}

\subsection{Markup of exponentiation}
We do not have any special requirements related to exponentiation, except that markup must be based on information given in \url{https://www.w3.org/TR/MathML3/chapter3.html#presm.msup}.

\subsection{Markup of vectors}
There are several ways to specify a vector in mathematical notation. One way is to place a right arrow over the symbol representing the variable, such as 
\[
\vec{v}\quad \text{and}\quad \overrightarrow{AB}
\]
Another way is to present the vector in some kind of bold font, as in 
\[
\mathbf{v} = v_x\;  \mathbf{i} + v_y\; \mathbf{j} + v_z\;  \mathbf{k}
\]

\bigskip
For vector notation with arrows we require the following markup:
\begin{kodeblokk}
\begin{verbatimtab}[3]
<mover>
	[a single element representing the vector]
	<mo>&#8594;</mo>
</mover>
\end{verbatimtab}
\end{kodeblokk}

\begin{eksempler}
The very simple expression $\vec{v}$ must be represented by
\begin{kodeblokk}
\begin{verbatimtab}[3]
<mover>
	<mo>v</mo>
	<mo>&#8594;</mo>
</mover>
\end{verbatimtab}
\end{kodeblokk}
 Note that the placeholder \markup{[a single element representing the vector]} may represent more complex notation. This means that $\vec{a} = \vec{a_1} + \vec{a_2}$ must be marked up as:
\begin{kodeblokk}
\begin{verbatimtab}[3]
<mover>
	<mi>a</mi>
	<mo>&#8594;</mo>
</mover>
<mo>=</mo>
<mover>
	<msub>
		<mi>a</mi>
		<mn>1</mn>
	</msub>
	<mo>&#8594;</mo>
</mover>
<mo>+</mo>
<mover>
	<msub>
		<mi>a</mi>
		<mn>2</mn>
	</msub>
	<mo>&#8594;</mo>
</mover>
\end{verbatimtab}
\end{kodeblokk}
\end{eksempler}

\medskip The required markup for vectors represented by a bold font, is to add the \markup{mathvariant} attribute to the \element{mi} element representing the vector. The attribute value \emph{must} be a string containing the substring \markup{bold}, This will typically mean one of \markup{bold}, \markup{bold-italic}, \markup{bold-sans-serif} or \markup{sans-serif-bold-italic}:
\begin{kodeblokk}
\begin{verbatimtab}[3]
<mi mathvariant="[string containing the substring 'bold']">
	[symbol representing the vector]
</mi>
\end{verbatimtab}
\end{kodeblokk}
This means that the expression
\[
\mathbf{v} = v_x\;  \mathbf{i} + v_y\; \mathbf{j} + v_z\;  \mathbf{k}
\]
can be marked up as
\begin{kodeblokk}
\begin{verbatimtab}[3]
<mi mathvariant="bold">v</mi>
<mo>=</mo>
<msub>
	<mi>v</mi>
	<mi>x</mi>
</msub>
<mo>&#8290;</mo>
<mi mathvariant="bold">i</mi>
<mo>+</mo>
<msub>
	<mi>v</mi>
	<mi>y</mi>
</msub>
<mo>&#8290;</mo>
<mi mathvariant="bold">j</mi>
<mo>+</mo>
<msub>
	<mi>v</mi>
	<mi>z</mi>
</msub>
<mo>&#8290;</mo>
<mi mathvariant="bold">k</mi>
\end{verbatimtab}
\end{kodeblokk}
Note that another attribute value than \markup{bold} could be used, in order to represent different types of bold font.

\innholdNyttBlokkSlutt
\subsection{Markup of fractions}
The MathML element \element{mfrac} must always be used for markup of fractions.

Even though the expression $1/2 + 1/2 =1$ could be marked up as
\begin{kodeblokk}
\begin{verbatimtab}[3]
<mn>1</mn>
<mo>/</mo>
<mn>2</mn>
<mo>+</mo>
<mn>1</mn>
<mo>/</mo>
<mn>2</mn>
<mo>=</mo>
<mn>1</mn>
\end{verbatimtab}
\end{kodeblokk}
we require the use of \element{mfrac} to represent the fractions:
\begin{kodeblokk}
\begin{verbatimtab}[3]
<mfrac bevelled="true">
	<mn>1</mn>
	<mn>2</mn>
</mfrac>
<mo>+</mo>
<mfrac bevelled="true">
	<mn>1</mn>
	<mn>2</mn>
</mfrac>
<mo>=</mo>
<mn>1</mn>
\end{verbatimtab}
\end{kodeblokk}
Note the use of the \markup{bevelled} attribute to separate the numerator and denominator  with a slash rather than with a horizontal line.

Of course, if a horizontal line \emph{is} required, as in this expression:
\[
x = \frac{1}{a + b}
\]
then the \markup{bevelled} attribute should not be used:
\begin{kodeblokk}
\begin{verbatimtab}[3]
<mi>x</mi>
<mo>=</mo>
<mfrac>
	<mn>1</mn>
	<mrow>
		<mi>a</mi>
		<mo>+</mo>
		<mi>b</mi>
	</mrow>
</mfrac>
\end{verbatimtab}
\end{kodeblokk}


\bigskip
Consult \url{https://www.w3.org/TR/MathML3/chapter3.html#presm.mfrac} for further information about use of the \element{mfrac} element. 



\innholdNyttBlokkStart{20190130}
%\subsection{Lower and upper indices}
\subsection{Lower indices}

%\subsubsection{Lower indices}
For lower indices, as in $A_T = A_1 + A_2$ we use the MathML element \element{msub}.

For a numeric index, the required markup is 
\begin{kodeblokk}
\begin{verbatimtab}[3]
<msub>
	<mi>
		[a single greek letter 
		or 
		a single letter in the reqions a-z or A-Z]
	</mi>
	<mn>[one or more integers]</mn>
</msub>
\end{verbatimtab}
\end{kodeblokk}
and for a symbolic index, the required markup is 
\begin{kodeblokk}
\begin{verbatimtab}[3]
<msub>
	<mi>
		[a single greek letter 
		or 
		a single letter in the reqions a-z or A-Z]
	</mi>
	<mi>
		[a single greek letter 
		or 
		a single letter in the reqions a-z or A-Z]
	</mi>
</msub>
\end{verbatimtab}
\end{kodeblokk}

\begin{eksempler}
Based on this,
	\[ A_T = A_1 + A_2 \]
	must be marked up as
\begin{kodeblokk}
\begin{verbatimtab}[3]
<msub>
	<mi>A</mi>
	<mi>T</mi>
</msub>
<mo>=</mo>
<msub>
	<mi>A</mi>
	<mn>1</mn>
</msub>
<mo>+</mo>
<msub>
	<mi>A</mi>
	<mn>2</mn>
</msub>
\end{verbatimtab}
\end{kodeblokk}
and
	\[ I_\alpha = \frac{I_\beta - I_\gamma}{2} \]
must be marked up as
\begin{kodeblokk}
\begin{verbatimtab}[3]
<msub>
	<mi>I</mi>
	<mi>&alpha;</mi>
</msub>
<mo>=</mo>
<mfrac>
	<mrow>
		<msub>
			<mi>I</mi>
			<mi>&beta;</mi>
		</msub>
		<mo>-</mo>
		<msub>
			<mi>I</mi>
			<mi>&gamma;</mi>
		</msub>
	</mrow>
	<mn>2</mn>
</mfrac>
\end{verbatimtab}
\end{kodeblokk}
	
\end{eksempler}


%\subsubsection{Upper indices}
%\todo{Spesifisere markup, og skrive transformasjonsregler}
%\subsubsection{Simultaneous lower and upper indices}
%\todo{Spesifisere markup, og skrive transformasjonsregler}
\innholdNyttBlokkSlutt
\subsection{Markup of functions with one argument}
A mathematical function, such as $f(x)$, $x(t)$, $F(x)$, $\psi (t)$ and similar, must be marked up as follows:
\begin{kodeblokk}
	\begin{verbatimtab}[3]
<mrow>
	<mi>
		[a single greek letter 
		or 
		a single letter in the reqions a-z or A-Z]
	</mi>
	<mo>&ApplyFunction;</mo>
	<mfenced>[any one child]</mfenced>
</mrow>
	\end{verbatimtab}
\end{kodeblokk}

\kravliste{
\item The markup of the function must be placed inside an \element{mrow} element. This \element{mrow} must contain exactly three children.
\item The first child of the \element{mrow} must be an \element{mi} element, containing one single Greek or single English letter, in upper or lower case.
\item The second child of the \element{mrow} must be an \element{mo} element, containing the \emph{Function Application Entity} \entitet{ApplyFunction}.
\item The third child of the \element{mrow} must be an \element{mfenced} element, containing the argument to the function. The \element{mfenced} element must have exactly one child. Apart from that, there are no requirements on the content of the \element{mfenced} element.
}

\begin{eksempler}
	The expression 
	\[g(x)\]
	shall be marked up as
	\begin{kodeblokk}
		\begin{verbatimtab}[3]
<mrow>
	<mi>g</mi>
	<mo>&ApplyFunction;</mo>
	<mfenced>
		<mi>x</mi>
	</mfenced>
</mrow>
		\end{verbatimtab}
	\end{kodeblokk}

And similar, the expression 
\[\psi(t)\]
shall be marked up as
\begin{kodeblokk}
\begin{verbatimtab}[3]
<mrow>
	<mi>&psi;</mi>
	<mo>&ApplyFunction;</mo>
	<mfenced>
		<mi>t</mi>
	</mfenced>
</mrow>
\end{verbatimtab}
\end{kodeblokk}

If the argument to the function is more complicated, such as in 
\[f(x + \Delta x)\] 
the corresponding markup will also be more complicated:
\begin{kodeblokk}
	\begin{verbatimtab}[3]
<mrow>
	<mi>f</mi>
	<mo>&ApplyFunction;</mo>
	<mfenced>
		<mrow>
			<mi>x</mi>
			<mo>+</mo>
			<mi>&Delta;</mi>
			<mi>x</mi>
		</mrow>
	</mfenced>
</mrow>
\end{verbatimtab}
\end{kodeblokk}

\end{eksempler}

\subsection{Markup of functions with two or more arguments}
A mathematical function with two or more arguments, such as $f(x,y,z)$, $F(x,t)$, $\psi (r, \theta )$ and similar, must be marked up as follows:
\begin{kodeblokk}
	\begin{verbatimtab}[3]
<mrow>
	<mi>
		[a single greek letter 
		or 
		a single letter in the reqions a-z or A-Z]
	</mi>
	<mo>&ApplyFunction;</mo>
	<mfenced>[two or more children]</mfenced>
</mrow>
	\end{verbatimtab}
\end{kodeblokk}

\kravliste{
	\item The markup of the function must be placed inside an \element{mrow} element. This \element{mrow} element must contain exactly three children.
	\item The first child of the \element{mrow} element must be an \element{mi} element, containing one single Greek or single English letter, in upper or lower case.
	\item The second child of the \element{mrow} element must be an \element{mo} element, containing the \emph{Function Application Entity} \entitet{ApplyFunction}.
	\item The third child of the \element{mrow} element must be an \element{mfenced} element, containing the arguments to the function. The number of children of the \element{mfenced} element must be equal to the number of arguments to the function. Apart from that, there are no requirements on the content of the \element{mfenced} element.
	
	Please observe that, when the \element{mfenced} element is rendered correctly, the comma separators are automatically inserted.
}

\begin{eksempler}
	The expression 
	\[f(x,y,z)\]
	shall be marked up as
	\begin{kodeblokk}
		\begin{verbatimtab}[3]
<mrow>
	<mi>f</mi>
	<mo>&ApplyFunction;</mo>
	<mfenced>
		<mi>x</mi>
		<mi>y</mi>
		<mi>z</mi>
	</mfenced>
</mrow>
		\end{verbatimtab}
	\end{kodeblokk}
	
	And the two-argument function 
	$\psi (r, \theta )$
	shall be marked up as
	\begin{kodeblokk}
		\begin{verbatimtab}[3]
<mrow>
	<mi>&psi;</mi>
	<mo>&ApplyFunction;</mo>
	<mfenced>
		<mi>r</mi>
		<mi>&theta;</mi>
	</mfenced>
</mrow>
		\end{verbatimtab}
	\end{kodeblokk}
\end{eksempler}


\subsection{Markup of named functions}\label{named-functions}
A known mathematical function, such as $\sin \alpha$, $\ln x$, $\arccos(x)$ and similar, must be marked up as follows:
\begin{kodeblokk}
\begin{verbatimtab}[3]
<mrow>
	<mi>[function name]</mi>
	<mo>&ApplyFunction;</mo>
	[any one element that represents the argument(s) to the function]
</mrow>
\end{verbatimtab}
\end{kodeblokk}

\kravliste{
	\item The markup of the function must be placed inside an \element{mrow} element. This \element{mrow} must contain exactly three children.
	\item The first child of the \element{mrow} must be an \element{mi} element, containing the name of the function.
	\item The second child of the \element{mrow} must be an \element{mo} element, containing the \emph{Function Application Entity} \entitet{ApplyFunction}.
	\item There are no particular requirements to the last element, except that it must correctly represent the argument(s) to the function.
}

\begin{eksempler}
	The expression 
	\[g(\alpha) = \sin \alpha\]
	shall be marked up as
\begin{kodeblokk}
\begin{verbatimtab}[3]
<mrow>
	<mi>g</mi>
	<mo>&ApplyFunction;</mo>
	<mfenced>
		<mi>&alpha;</mi>
	</mfenced>
</mrow>
<mo>=</mo>
<mrow>
	<mi>sin</mi>
	<mo>&ApplyFunction;</mo>
	<mi>&alpha;</mi>
</mrow>
\end{verbatimtab}
\end{kodeblokk}

And, the expression 
\[
\ln (x\, y) = \ln x + \ln y
\]
shall be marked up as
\begin{kodeblokk}
\begin{verbatimtab}[3]
<mrow>
	<mi>ln</mi>
	<mo>&ApplyFunction;</mo>
	<mfenced>
		<mrow>
			<mi>x</mi>
			<mo>&InvisibleTimes;</mo>
			<mi>y</mi>
		</mrow>
	</mfenced>
</mrow>
<mo>=</mo>
<mrow>
	<mi>ln</mi>
	<mo>&ApplyFunction;</mo>
	<mi>x</mi>
</mrow>
<mo>+</mo>
<mrow>
	<mi>ln</mi>
	<mo>&ApplyFunction;</mo>
	<mi>y</mi>
</mrow>
\end{verbatimtab}
\end{kodeblokk}	
\end{eksempler}


\innholdNyttBlokkStart{20190130}
Note that, for an expression on the form $\cos 2 x$, the following markup is \textbf{NOT} correct:
\begin{kodeblokk}
	\begin{verbatimtab}[3]
<mrow>
	<mi>cos</mi>
	<mo>&#8289;</mo>
	<mn>2</mn>
	<mo>&#8290;</mo>
	<mi>x</mi>
</mrow>
	\end{verbatimtab}
\end{kodeblokk}	
This markup breaks the requirement that the \element{mrow} element must contain exactly three children. Instead, the markup must be as follows:
\begin{kodeblokk}
	\begin{verbatimtab}[3]
<mrow>
	<mi>cos</mi>
	<mo>&#8289;</mo>
	<mrow>
		<mn>2</mn>
		<mo>&#8290;</mo>
		<mi>x</mi>
	</mrow>
</mrow>
\end{verbatimtab}
\end{kodeblokk}	

\innholdNyttBlokkSlutt

\subsection{Markup of limits and the derivative}

The limit of a mathematical expression must generically be marked up as follows:
\begin{kodeblokk}
\begin{verbatimtab}[3]
<mrow>
	<munder>
		<mo>lim</mo>
		[element representing the conditions]
	</munder>
	<mo>&ApplyFunction;</mo>
	[one element representing some mathematcal function or expression]
</mrow>
\end{verbatimtab}
\end{kodeblokk}	

\kravliste{
	\item The construction must be placed inside an \element{mrow} element. This \element{mrow} must contain exactly three children.
	\item The first child of the \element{mrow} element  must be an \element{munder} element. This \element{munder} element must contain exactly two elements.
	\begin{itemize}
		\item The first child of the \element{munder} element must be an \element{mo} element containing the text \markup{lim}.
		\item The second child of the \element{munder} element must represent the condition for finding the limit. If necessary, place the condition inside an \element{mrow} element to represent it as one element.
	\end{itemize}
	\item The second child of the \element{mrow} must be an \element{mo} element, containing the \emph{Function Application Entity} \entitet{ApplyFunction}.
	\item There are no particular requirements to the last element, except that it must correctly represent the expression to find the limit of. If necessary, place the expression inside an \element{mrow} element to represent it as one element.
}

\begin{eksempler}
	
	The expression 
\[
g(x) =\lim_{t \rightarrow T} f(x,t)
\]
shall be marked up as
\begin{kodeblokk}
	\begin{verbatimtab}[3]
<mrow>
	<mi>g</mi>
	<mo>&ApplyFunction;</mo>
	<mfenced>
		<mi>x</mi>
	</mfenced>
</mrow>
<mo>=</mo>
<mrow>
	<munder>
		<mo>lim</mo>
		<mrow>
			<mi>t</mi>
			<mo>&rarr;</mo>
			<mi>T</mi>
		</mrow>
	</munder>
	<mo>&ApplyFunction;</mo>
	<mrow>
		<mi>f</mi>
		<mo>&ApplyFunction;</mo>
		<mfenced>
			<mi>x</mi>
			<mi>t</mi>
		</mfenced>
	</mrow>
</mrow>
	\end{verbatimtab}
\end{kodeblokk}

	And, the expression 
	\[
	\lim_{x \rightarrow 0} \frac{1}{x} = \infty
	\]
	shall be marked up as
	\begin{kodeblokk}
		\begin{verbatimtab}[3]
<mrow>
	<munder>
		<mo>lim</mo>
		<mrow>
			<mi>x</mi>
			<mo>&rarr;</mo>
			<mn>0</mn>
		</mrow>
	</munder>
	<mo>&ApplyFunction;</mo>
	<mfrac>
		<mn>1</mn>
		<mi>x</mi>
	</mfrac>
</mrow>
<mo>=</mo>
<mi>&infin;</mi>		
\end{verbatimtab}
	\end{kodeblokk}
\end{eksempler}

\bigskip
There are several ways to write the derivative of a mathematical expression. One is to add a prime symbol after the function or expression, as in 
\[
f'(x)\quad\text{and}\quad (\sin x)'
\]
This is \emph{Lagrange's notation}. An alternative is to use \emph{Leibniz's notation}:
\[
\frac{\text{d}f(x)}{\text{d}x}\quad\text{and}\quad \frac{\text{d}\sin x}{\text{d}x}
\]
There are several more ways to write the derivative, but these are the most common, so we will focus on these two.

\subsubsection{The \emph{Lagrange notation}}
When we want to write the derivative of a function using \emph{Lagrange's notation}, as in $f'(x)$, the required markup is

\begin{kodeblokk}
\begin{verbatimtab}[3]
<mrow>
	<msup>
		<mi>
			[a single greek letter 
			or 
			a single letter in the reqions a-z or A-Z]
		</mi>
		<mo>&prime;</mo>
	</msup>
	<mo>&ApplyFunction;</mo>
	<mfenced>[markup of one or more arguments]</mfenced>
</mrow>
\end{verbatimtab}
\end{kodeblokk}

\kravliste{
	\item The markup of the derivative of a function must be placed inside an \element{mrow} element. This \element{mrow} must contain exactly three children.
	\item The first child of the \element{mrow} element must be an \element{msup} element.
	\begin{itemize}
		\item The first child of the \element{msup} element must be an \element{mi} element, containing one single Greek or single English letter, in upper or lower case.
		\item The second child of the \element{msup} element must be an \element{mo} element, containing the entity for the prime symbol, \entitet{prime}.
		
		\textbf{Note:} As an alternative to the \entitet{prime} entity, we also allow use of the \entitet{Prime} entity (note the capital~P) to represent the second derivative, and the \entitet{tprime} to represent the third derivative.
	\end{itemize}
	
	\item The second child of the \element{mrow} element  must be an \element{mo} element, containing the \emph{Function Application Entity} \entitet{ApplyFunction}.
	\item The third child of the \element{mrow} element must be an \element{mfenced} element, containing the argument(s) to the function. 
}

\begin{eksempler}
The definition of the derivative of a function $f$ is often given as
\[
f'(x) = \lim_{h \rightarrow 0} \frac{f(x+h) - f(h)}{h}
\]
The correct markup of this expression is 
\begin{kodeblokk}
\begin{verbatimtab}[3]
<mrow>
	<msup>
		<mi>f</mi>
		<mo>&prime;</mo>
	</msup>
	<mo>&ApplyFunction;</mo>
	<mfenced>
		<mi>x</mi>
	</mfenced>
</mrow>
<mo>=</mo>
<mrow>
	<munder>
		<mo>lim</mo>
		<mrow>
			<mi>h</mi>
			<mo>&rarr;</mo>
			<mn>0</mn>
		</mrow>
	</munder>
	<mo>&ApplyFunction;</mo>
	<mfrac>
		<mrow>
			<mrow>
				<mi>f</mi>
				<mo>&ApplyFunction;</mo>
				<mfenced>
					<mrow>
						<mi>x</mi>
						<mo>+</mo>
						<mi>h</mi>
					</mrow>
				</mfenced>
			</mrow>
			<mo>-</mo>
			<mrow>
				<mi>f</mi>
				<mo>&ApplyFunction;</mo>
				<mfenced>
					<mi>x</mi>
				</mfenced>
			</mrow>
		</mrow>
		<mi>h</mi>
	</mfrac>
</mrow>
\end{verbatimtab}
\end{kodeblokk}
And
\[
f''(x) = f'(f'(x))
\]
must be marked up as
\begin{kodeblokk}
\begin{verbatimtab}[3]
<mrow>
	<msup>
		<mi>f</mi>
		<mo>&Prime;</mo>
	</msup>
	<mo>&ApplyFunction;</mo>
	<mfenced>
		<mi>x</mi>
	</mfenced>
</mrow>
<mo>=</mo>
<mrow>
	<msup>
		<mi>f</mi>
		<mo>&prime;</mo>
	</msup>
	<mo>&ApplyFunction;</mo>
	<mfenced>
		<mrow>
			<msup>
				<mi>f</mi>
				<mo>&prime;</mo>
			</msup>
			<mo>&ApplyFunction;</mo>
			<mfenced>
				<mi>x</mi>
			</mfenced>
		</mrow>
	</mfenced>
</mrow>
\end{verbatimtab}
\end{kodeblokk}

\end{eksempler}

\bigskip
When we want to write the derivative of a named function using \emph{Lagrange's notation}, as in $(\sin x)'$, the required markup is

\begin{kodeblokk}
\begin{verbatimtab}[3]
<msup>
	<mfenced>
		[markup representing the function]
	</mfenced>
	<mo>&prime;</mo>
</msup>
\end{verbatimtab}
\end{kodeblokk}

\kravliste{
	\item The markup of the derivative of a named function must be placed inside an \element{msup} element.
%	\begin{itemize}
		\item The first child of the \element{msup} element must be an \element{mfenced} element, and that \element{mfenced} element must contain the markup of the function.
		\item The second child of the \element{msup} element must be an \element{mo} element, containing the entity for the prime symbol, \entitet{prime}.
		
		\textbf{Note:} As an alternative to the \entitet{prime} entity, we also allow use of the \entitet{Prime} entity to represent the second derivative, and the \entitet{tprime} to represent the third derivative.
%	\end{itemize}
}

\begin{eksempler}
The expression 
\[
(3 a x^3)'' = (9 ax^2)' = 18 a x
\]
must be marked up as
\begin{kodeblokk}
\begin{verbatimtab}[3]
<msup>
	<mfenced>
		<mrow>
			<mn>3</mn>
			<mo>&it;</mo>
			<mi>a</mi>
			<mo>&it;</mo>
			<msup>
				<mi>x</mi>
				<mn>3</mn>
			</msup>
		</mrow>
	</mfenced>
	<mo>&Prime;</mo>
</msup>
<mo>=</mo>
<msup>
	<mfenced>
		<mrow>
			<mn>9</mn>
			<mo>&it;</mo>
			<mi>a</mi>
			<mo>&it;</mo>
			<msup>
				<mi>x</mi>
				<mn>2</mn>
			</msup>
		</mrow>
	</mfenced>
	<mo>&prime;</mo>
</msup>
<mo>=</mo>
<mn>18</mn>
<mo>&it;</mo>
<mi>a</mi>
<mo>&it;</mo>
<mi>x</mi>
\end{verbatimtab}
\end{kodeblokk}


\end{eksempler}

\subsubsection{The \emph{Leibniz notation}}
When we want to write the derivative of a function using \emph{Leibniz's notation}, as in
\[
\frac{\text{d}f(x)}{\text{d}x}
\]
the required markup is

\begin{kodeblokk}
\begin{verbatimtab}[3]
<mfrac>
	<mrow>
		<mo>&DifferentialD;</mo>
		[markup required to represent the function]
	</mrow>
	<mrow>
		<mo>&DifferentialD;</mo>
		<mi>
			[a letter in the region a-z]
		</mi>
	</mrow>
</mfrac>
\end{verbatimtab}
\end{kodeblokk}

\kravliste{
	\item The markup of the derivative of a function, using \emph{Leibniz's notation,} must be placed inside an \element{mfrac} element.
	\item The first child of the \element{mfrac} element must be an \element{mrow} element.
	\begin{itemize}
		\item The first child of this \element{mrow} element must be an \element{mo} element, containing the entity for the \emph{differential d} symbol, \entitet{DifferentialD} (alternative short form: \entitet{dd} \innholdNyttInline{20190130}{or numeric form \entitet{8518}}).
		\item The rest of this \element{mrow} element must be filled up with the necessary markup to represent the function.
	\end{itemize}
	\item The second child of the \element{mfrac} element must also be an \element{mrow} element, with only two children.
	\begin{itemize}
		\item The first child of this second \element{mrow} element must again be an \element{mo} element, also this one containing the entity for the \emph{differential d} symbol, \entitet{DifferentialD}.
		\item The second (and last) child of the \element{mrow} element must be an \element{mi} element, containg a single letter in the region ''a to 'z''.
	\end{itemize}
}

\begin{eksempler}
The expression
\[
f'(x) = \frac{\text{d}f(x)}{\text{d}x}
\]
must be marked up as
\begin{kodeblokk}
\begin{verbatimtab}[3]
<mrow>
	<msup>
		<mi>f</mi>
		<mo>&prime;</mo>
	</msup>
	<mo>&ApplyFunction;</mo>
	<mfenced>
		<mi>x</mi>
	</mfenced>
</mrow>
<mo>=</mo>
<mfrac>
	<mrow>
		<mo>&DifferentialD;</mo>
		<mrow>
			<mi>f</mi>
			<mo>&ApplyFunction;</mo>
			<mfenced>
				<mi>x</mi>
			</mfenced>
		</mrow>
	</mrow>
	<mrow>
		<mo>&dd;</mo>
		<mi>x</mi>
	</mrow>
</mfrac>
\end{verbatimtab}
\end{kodeblokk}
\end{eksempler}

Markup of higher order derivatives of a function, based on \emph{Leibniz's notation}, is just an extension of the markup of the first derivative:
\begin{kodeblokk}
\begin{verbatimtab}[3]
<mfrac>
	<mrow>
		<msup>
			<mo>&DifferentialD;</mo>
			<mn>
				[order of differentiation]
			</mn>
		</msup>
		[markup required to represent the function]
	</mrow>
	<mrow>
		<mo>&dd;</mo>
		<msup>
			<mi>
				[a letter in the region a-z]
			</mi>
			<mn>
				[order of differentiation]
			</mn>
		</msup>
	</mrow>
</mfrac>
\end{verbatimtab}
\end{kodeblokk}

\kravliste{
	\item The markup of the second derivative of a function, using \emph{Leibniz's notation,} must be placed inside an \element{mfrac} element.
	\item The first child of the \element{mfrac} element must be an \element{mrow} element.
	\begin{itemize}
		\item The first child of this \element{mrow} element must be an \element{msup} element
		\begin{itemize}
			\item The first child of this \element{msup} element must be an \element{mo} element, containing the entity for the \emph{differential d} symbol, \entitet{DifferentialD}, with alternative short form \entitet{dd} \innholdNyttInline{20190130}{or numeric form \entitet{8518}}.
			\item The second child of this \element{msup} element must be an \element{mn} element, containing the number representing the order of differentiation.
		\end{itemize}
		\item The rest of this \element{mrow} element must be filled up with the necessary markup to represent the function.
	\end{itemize}
	\item The second child of the \element{mfrac} element must also be an \element{mrow} element, with only two children.
	\begin{itemize}
		\item The first child of this second \element{mrow} element must again be an \element{mo} element, also this one containing the entity for the \emph{differential d} symbol, \entitet{DifferentialD}.
		\item The second (and last) child of the \element{mrow} element must be an \element{msup} element
		\begin{itemize}
			\item The first child of this \element{msup} element must  be an \element{mi} element, containg a single letter in the region ''a to ''z''.
			\item The second child of this \element{msup} element must be an \element{mn} element, containing the number representing the order of differentiation.
		\end{itemize}
	\end{itemize}
}

\begin{eksempler}
The expression
\[
g''(x) =\frac{\text{d}^2 \sin x}{\text{d} x^2}
\]
must be marked up as
\begin{kodeblokk}
\begin{verbatimtab}[3]
<mrow>
	<msup>
		<mi>g</mi>
		<mo>&Prime;</mo>
	</msup>
	<mo>&ApplyFunction;</mo>
	<mfenced>
		<mi>x</mi>
	</mfenced>
</mrow>
<mo>=</mo>
<mfrac>
	<mrow>
		<msup>
			<mo>&DifferentialD;</mo>
			<mn>2</mn>
		</msup>
		<mrow>
			<mi>sin</mi>
			<mo>&ApplyFunction;</mo>
			<mi>x</mi>
		</mrow>
	</mrow>
	<mrow>
		<mo>&dd;</mo>
		<msup>
			<mi>x</mi>
			<mn>2</mn>
		</msup>
	</mrow>
</mfrac>
\end{verbatimtab}
\end{kodeblokk}

\end{eksempler}


\innholdNyttBlokkStart{20190130}
\subsection{Markup of the integral}
The following entities are relevant for markup of integrals:

\begin{tabular}{clll}
	\multicolumn{1}{c}{\textbf{Symbol}}
	& \multicolumn{1}{c}{\textbf{Entity}}
	& \multicolumn{1}{c}{\textbf{Numeric}}
	& \multicolumn{1}{c}{\textbf{Description}}\\
	$\int$ & \entitet{int} & \entitet{\#8747} & Integral\\
	$\iint$ & \entitet{Int} & \entitet{\#8748} & Double integral\\
	$\iiint$ & \entitet{tint} & \entitet{\#8749} & Triple integral\\
%	$\iiiint$ & \entitet{qint} & \entitet{\#10764} & Quadruple integral\\
	$\oint$ & \entitet{conint} & \entitet{\#8750} & Contour integral
%	$\varointclockwise$ & \entitet{cwconint} & \entitet{\#8754} & Clockwise contour integral\\
%	$\oiint$ & \entitet{Conint} & \entitet{\#8751} & Surface integral\\
%	$\oiiint$ & \entitet{Conint} & \entitet{\#8752} & Volume integral
\end{tabular}\label{integraltegn}

For the \emph{indefinite integral}, with a very simple argument, such as $\int f = g +C$ and $T = \oint f$ and even $a(x) = \iint \cos x$, the following markup will be sufficient:
\begin{kodeblokk}
\begin{verbatimtab}[3]
<mo>&int;</mo>
<mi>f</mi>
<mo>=</mo>
<mi>g</mi>
<mo>+</mo>
<mi>C</mi>
\end{verbatimtab}
\end{kodeblokk}
and
\begin{kodeblokk}
\begin{verbatimtab}[3]
<mi>T</mi>
<mo>=</mo>
<mo>&#8750;</mo>
<mi>g</mi>	
\end{verbatimtab}
\end{kodeblokk}
and
\begin{kodeblokk}
\begin{verbatimtab}[3]
<mrow>
	<mi>a</mi>
	<mo>&ApplyFunction;</mo>
	<mfenced>
		<mi>x</mi>
	</mfenced>
</mrow>
<mo>=</mo>
<mo>&#8748;</mo>
<mrow>
	<mi>cos</mi>
	<mo>&ApplyFunction;</mo>
	<mi>x</mi>
</mrow>
\end{verbatimtab}
\end{kodeblokk}
respectively.

\bigskip
However, once the expressions get a little more complex, the requirements get more strict. Just the need to specify the integration variable, as in 
\[
\int f(x) \text{d}x
\]
will require a very specific markup:
\begin{kodeblokk}
	\begin{verbatimtab}[3]
<mo>&int;</mo>
[markup of function to be integrated
<mrow>
	<mo>&dd;</mo>
	<mi>[integration variable]</mi>
</mrow>
\end{verbatimtab}
\end{kodeblokk}
Note that the \entitet{int} entity in the first \element{mo} element could just as well be any of the other integral entities listed \vpageref{integraltegn}.

So the complete markup of 
\[
C(z) = \oint \frac{1}{z}\text{d}z
\]
would be
\begin{kodeblokk}
\begin{verbatimtab}[3]
<mrow>
	<mi>C</mi>
	<mo>&af;</mo>
	<mfenced>
		<mi>z</mi>
	</mfenced>
</mrow>
<mo>=</mo>
<mo>&oint;</mo>
<mfrac>
	<mn>1</mn>
	<mi>z</mi>
</mfrac>
<mrow>
	<mo>&dd;</mo>
	<mi>z</mi>
</mrow>
\end{verbatimtab}
\end{kodeblokk}

\bigskip
For \emph{definite integrals}, where the interval of the integration is specified, there are further requirements to the markup.
\begin{kodeblokk}
\begin{verbatimtab}[3]
<mrow>
	<munderover>
		<mo>[integration sign]</mo>
		[upper limit]
		[lower limit]
	</munderover>
	[any markup representing the expression to be integrated]
	<mrow>
		<mo>&dd;</mo>
		<mi>[a letter in the region a-z]</mi>
	</mrow>
</mrow>
\end{verbatimtab}
\end{kodeblokk}

%
\kravliste{
	\item The whole expression must be placed inside an \element{mrow} element.
	\item This \element{mrow} must have three children:
	\begin{itemize}
		\item First, an \element{munderover} element, with exactly three children:
		\begin{itemize}
			\item The first child of this \element{munderover} element must be an \element{mo} element containing one of the entities listed \vpageref{integraltegn}.
			\item The second child must be an element containing the necessary markup to represent the lower limit of the integration interval.
			\item The last child must be an element containing the necessary markup to represent the upper limit of the integration interval.
		\end{itemize}
		\item Second, any element containing the markup of the expression to be integrated.
		\item Third, an \element{mrow} element, containing exactly two children:
		\begin{itemize}
			\item The first child of this \element{mrow} element must be an \element{mo} element, also this one containing the entity for the \emph{differential d} symbol, \entitet{DifferentialD}.
			\item The second child of this \element{mrow} element must  be an \element{mi} element, containing a single letter in the region ''a to ''z''.
		\end{itemize}
	\end{itemize}
}

\begin{eksempler}
	The expression
	\[
	F(x) = \int_{t=a}^{t=x} f(t) \text{d}t
	\]
must be marked up as
\begin{kodeblokk}
\begin{verbatimtab}[3]
<mrow>
	<mi>F</mi>
	<mo>&ApplyFunction;</mo>
	<mfenced>
		<mi>x</mi>
	</mfenced>
</mrow>
<mo>=</mo>
<mrow>
	<munderover>
		<mo>&int;</mo>
		<mrow>
			<mi>t</mi>
			<mo>=</mo>
			<mi>a</mi>
		</mrow>
		<mrow>
			<mi>t</mi>
			<mo>=</mo>
			<mi>x</mi>
		</mrow>
	</munderover>
	<mrow>
		<mi>f</mi>
		<mo>&ApplyFunction;</mo>
		<mfenced>
			<mi>t</mi>
		</mfenced>
	</mrow>
	<mrow>
		<mo>&dd;</mo>
		<mi>t</mi>
	</mrow>
</mrow>
\end{verbatimtab}
\end{kodeblokk}

Another example:
\[
\oint \frac{1}{z}\text{d} z =\int_{0}^{2\pi} \frac{1}{e^{i t}}\, i\; e^{i t} \text{d}t
\]
must be represented by the following markup:
\begin{kodeblokk}
\begin{verbatimtab}[3]
<mo>&oint;</mo>
<mfrac>
	<mn>1</mn>
	<mi>z</mi>
</mfrac>
<mrow>
	<mo>&dd;</mo>
	<mi>z</mi>
</mrow>
<mo>=</mo>
<mrow>
	<munderover>
		<mo>&int;</mo>
		<mn>0</mn>
		<mrow>
			<mn>2</mn>
			<mo>&it;</mo>
			<mi>&pi;</mi>
		</mrow>
	</munderover>
	<mrow>
		<mfrac>
			<mn>1</mn>
			<msup>
				<mi>e</mi>
				<mrow>
					<mi>i</mi>
					<mo>&it;</mo>
					<mi>t</mi>
				</mrow>
			</msup>
		</mfrac>
		<mo>&it;</mo>
		<mi>i</mi>
		<mo>&it;</mo>
		<msup>
			<mi>e</mi>
			<mrow>
				<mi>i</mi>
				<mo>&it;</mo>
				<mi>t</mi>
			</mrow>
		</msup>
	</mrow>
	<mrow>
		<mo>&dd;</mo>
		<mi>t</mi>
	</mrow>
</mrow>
\end{verbatimtab}
\end{kodeblokk}

\end{eksempler}

\innholdNyttBlokkSlutt
\end{document}
